\documentclass[12pt,a4paper]{exam}
\usepackage[utf8]{inputenc}
\usepackage{amsmath}
\usepackage{amsfonts}
\usepackage{amssymb}
\usepackage{graphicx}

\newcommand\textbox[1]{%
	\parbox{.333\textwidth}{#1}%
}





% Customised items
\usepackage[shortlabels]{enumitem}

% Times New Roman
\usepackage[T1]{fontenc}
\usepackage{newtxmath,newtxtext}
% Customised items
\usepackage[shortlabels]{enumitem}

% Times New Roman
\usepackage[T1]{fontenc}
\usepackage{newtxmath,newtxtext}

% Page Setup
\usepackage[a4paper,
	bindingoffset=0.2in,%
	left=0.5in,
	right=0.5in,
	top=1in,
	bottom=1in,%
	footskip=.25in]{geometry}
% Code
\usepackage{courier} %% Sets font for listing as Courier.
\usepackage{listings, xcolor}
\lstset{
	tabsize = 4, %% set tab space width
	showstringspaces = false, %% prevent space marking in strings, string is defined as the text that is generally printed directly to the console
	numbers = left, %% display line numbers on the left
	commentstyle = \color{red}, %% set comment color
	keywordstyle = \color{blue}, %% set keyword color
	stringstyle = \color{red}, %% set string color
	rulecolor = \color{black}, %% set frame color to avoid being affected by text color
	basicstyle = \small \ttfamily , %% set listing font and size
	breaklines = true, %% enable line breaking
	numberstyle = \tiny,
}
\author{Yonten Jamtsho}
\begin{document}
	\begin{center}
		[This question paper contains \textbf{\textit{two}} printed pages]
	\end{center}
	
	\flushleft \textbf{BSc(IT)/2020/I/F/ITP102/I} 	\hfill \textbf{Student No: .................................}
	
	
	% Title
	\begin{center}
		\textbf{Semester End Examination, Spring 2021} \\
		\textbf{Royal University of Bhutan} \\
		\textbf{Gyalpozhing College of Information Technology} \\
		\textbf{BSc in Information Technology} - \textbf{Year I, Semester I} \\
		\textbf{ITP102 : Object Oriented Programming Fundamentals} \\
		
	\end{center}

	\flushleft \textbf{Time: 3 Hours} \hfill \textbf{Max. Marks: 50}
	\vspace{0.3cm}
	
	\hrule
	\vspace{0.1cm}
	\textit{Write your Student number on the top immediately on receipt of this question paper. All questions are compulsory, and marks are given at the end of each question. Parts of a question should be answered together. Spend the first 10 minutes in reading the questions.}
	\vspace{0.1cm}
	\hrule
	
	\vspace{0.2cm}
    
    % Part A title
    \begin{center}
    	\noindent \textbf{PART - A} \textbf{ [5 Marks]}\\
    	\noindent \textit{Answer all the questions} 
    \end{center}
    
    % Keep space between PAT A and MCQ
    \vspace{0.5cm}
    
    % MQC Title
    \noindent \textbf{Multiple Choice Questions} \hfill \textbf{[5 x 1 = 5]}
    
    % Begin your question with customized numbered list
    \begin{enumerate}[start=1,label={\bfseries Q\arabic*)}]
    	% Reduce the space between items
    	\itemsep0.2em
    	\item What are the relationships between the below-mentioned classes?\\
    	\textit{Vehicle, Suzuki, Hyundai}
    	\item[] 
    	\begin{oneparchoices}
    		\choice All three are super classest % \hspace is used to keep space between choice
    		\choice All three are sub classes 
    		\choice Suzuki is the super class, Vehicle and Hyundai are the subclass of Suzuki
    		\choice Vehicle is the super class, Suzuki and Hyundai are the sub class of Vehicle
    	\end{oneparchoices}
    	
    	\item  How would you declare a variable storing average grade?
    	\item[] 
    	\begin{oneparchoices}
    		\choice double grade = "30.0"; % \hspace is used to keep space between choice
    		\choice double grade = 30.0; 
    		\choice int grade = "30.0";
    		\choice int grade = 30.0;
    	\end{oneparchoices}
    	
    
    	\item What value is placed in x?\\
    	\textit{x = 60 > 100 ? 0 : 1;}
    	\item[]   
    	\begin{oneparchoices}
    		\choice  0% \hspace is used to keep space between choice
    		\choice  100
    		\choice 1
    		\choice 60
    	\end{oneparchoices}
    	
    	\item  What is the output of the given code? 
    	\begin{lstlisting}[language=Java][fragile]
    		class Output{
    			public static void main(String[] args) {
    				int a = 1;
    				int b = 2;
    				int c; int d;
    				c = ++b;
    				d = a++;
    				c++;
    				b++;
    				++a;
    				System.out.println(a + " " + " " + b + " " + c);
    			}
    		}
    	\end{lstlisting}
    	\item[]   
    	\begin{oneparchoices}
    		\choice  3 4 4% \hspace is used to keep space between choice
    		\choice  3 2 2 
    		\choice 2 4 4
    		\choice 4 4 4
    	\end{oneparchoices}
    	
    	\item In character stream I/O, a single read/write operation performs
    	\item[]   
    	\begin{oneparchoices}
    		\choice  Two bytes read/write at a time.% \hspace is used to keep space between choice
    		\choice  Eight bytes read/write at a time.
    		\choice One byte read/write at a time.
    		\choice Five bytes read/ write at a time.
    	\end{oneparchoices}
    \end{enumerate}
    
   
    \vspace{0.1mm}
    % Part B title
    \begin{center}
    	\textbf{PART - B} \textbf{[15 Marks]}\\
    	\noindent \textit{SHORT ANSWER QUESTION} \\
    	\noindent \textit{Answer all the questions} 
    \end{center}
    
    \begin{enumerate}[start=1,label={\bfseries Q\arabic*)}]
    
    	\item What is Error in Java. Give Example \hfill \textbf{[1.5]}
    	\item List with example, the methods to print the exception Messages. \hfill \textbf{[3]}
    	\item Write one difference between Testing Debugging.  \hfill \textbf{[1]}
    	\item When do we use \textbf{this}  and \textbf{super} keyword in java. Explain with code snippet.\hfill \textbf{[2]}
    	\item Write down three differences between abstract class and interface. \hfill \textbf{[3]}
    	\item What is Encapsulation in Java. Explain how Java achieves Encapsulation Principle. \hfill \textbf{[1.5]}
    	\item What is Command Line Argument.\hfill \textbf{[0.5]}
    	\item Draw the Java Architecture diagram. \hfill \textbf{[1.5]}
    \end{enumerate}
    \pagebreak
    % Part C title
    \begin{center}
    	\textbf{PART - III} \textbf{[25 Marks]}\\
    	\noindent \textit{LONG ANSWER QUESTIONS} \\
    	\noindent \textit{Answer all the questions}  
    \end{center}
    
    \begin{enumerate}[start=1,label={\bfseries Q\arabic*)}]
    	\item Create a custom exception called \textbf{MyException} and use \textbf{MyException} in a class \textbf{MyException1} and print the exception. Remember, your customException should have toString() method.\hfill\textbf{[5]}  	
    	\item Finding, \textbf{a} power \textbf{b}.  Write a java program that calculates \textbf{a} power \textbf{b}.  The \textbf{a} can be either int or float. The \textbf{b} can only be an int. Use constructor overloading and method overloading to calculate \textbf{power (a, b)}.\\
    	Example Input and Output:\\
    	power(2.5, 2) -> 6.25 \\power(25, 2) -> 625\\
    	\textbf{Note: Do not use inbuilt function pow.}  \hfill\textbf{[5]}
    	\item  \textbf{ListADT} has the following methods. \\
    	
    	Methods:
    	\begin{enumerate}
    		\item  \textbf{ListADT()}: Constructor
    		\item \textbf{add(element)}: Takes an int element to add into Array. The return type is void.
    		\item \textbf{contains(element)}: Takes an int element and returns whether it is present in the Array or not. The return type is boolean.
    		\item \textbf{get(index)}: Takes an int index and returns the element of that index. The return type is int.
    		\item \textbf{size()}: Returns the size of Array. The return type is int.
    		\item \textbf{toString()}: Converts the given input into the String format
    	\end{enumerate}
    	Design an interface where this can be used for various applications and create a class \textbf{ListADTClass} which implements this interface. 
    	\hfill\textbf{[10]} 
    	\item Create a class \textbf{OpenFileExample} to implement opening and reading a file using FileReader class. Do not forget to handle exception. Assume your own location of file in the computer.\hfill\textbf{[5]}
    	
    	
    	
    	$------------ ------BEST WISHES----------------------------$
    \end{enumerate}
    
    
\end{document}